\documentclass[12pt,a4paper]{scrartcl}
\usepackage{url, booktabs,pdflscape}
\usepackage[final]{pdfpages}
\usepackage[colorlinks]{hyperref}
\title{Student Robotics Kickstart Risk Assessment Form}

\begin{document}
\maketitle

\begin{description}
\item[Activity being assessed:] General Use of Student Robotics' Kit
\item[Location:] Schools, Colleges, etc.
\item[Who is exposed to the hazard:] Competitors, Teachers, Mentors
\end{description}

\begin{description}
\item[Assessor's name:]
\item[Assessor's job title:]
\item[Assessor's signature:]
\item[Date of assessment:]
\end{description}
\clearpage

\newcommand{\risk}[3]{
 #1 & #2 & #3 \\
}

\begin{landscape}
\section{Risks}
The following risks have been considered for unsupervised use of Student
Robotics Kit.  Further description of the meaning of risk ratings (presented in this section as $L \times S$) can be found in the next section.
\\
A safety briefing about use of Student Robotics kit is given at Kickstart.
This safety information, and instructions on use of our kit is also available on
our website (\url{https://www.studentrobotics.org/docs/kit}). It is advised that
a responsible adult supervises use of Student Robotics kit when used in school.
\\
This risk assessment does not cover construction activities such as metalwork
or woodwork, which should be covered by institution-specific risk assessments.

\bigskip
\begin{tabular*}{\linewidth}[c]{p{14em}p{30em}c}
\toprule
\textbf{Hazard} & \textbf{Control Measures} & \textbf{Risk Rating} \\
\midrule

\risk{Misuse of Batteries}
{If mishandled/misused, the batteries can explode or catch fire. 
\begin{itemize}
\item \textbf{Charging} Clear instructions on battery charging and safe battery
handling are available at our website. To mitigate risk of fire during charging
and in storage, fireproof bags are provided.
\item \textbf{Damage} It is advised any batteries showing signs of damage,
particularly bulging, are disposed of.
\item \textbf{Shorting} To minimise the risk of short-circuiting battery
terminals, it is advised no metallic tools are used to separate battery and
robot. In case of shorting, batteries are provided with an in-line fuse to
prevent explosion/fire, but the terminals may spark.
\item \textbf{Use} When testing robots, batteries need to be safely and securely
attached to the robot, in such a way that it is protected from flying debris.
\end{itemize}}
{4}

\risk{Irritation due to loose fibres in Charging Bags}
{Some charging bags distributed with this years kit, used to mitigate risk of
fire during storage and charging, contain loose fibreglass fibres which may
cause minor skin irritation. Use of gloves is advised when using these bags.}
{2}

\risk{Interaction with robots: electric shock, minor injury}
{Minor injury may be caused when testing unfinished/prototype robots which may
not meet Student Robotics' safety guidelines. It is strongly advised that for
any testing, the power switch is easily accessible, and that the battery is
safely and securely attached to the robot. Before testing a robot, all occupants
of the room should be made aware.}
{1}

\bottomrule
\end{tabular*}
\end{landscape}

\begin{landscape}

\section{Assessment Guidance}

The risk ratings of the risks in the previous section are calculated by multiplying $L$, the likelihood rating, by $S$, the severity rating.

\bigskip
\begin{minipage}[b]{0.5\linewidth}
\begin{tabular}[c]{lc}
  \toprule
  \textbf{Likelihood} & \textbf{Likelihood rating} \\
  \midrule
  Very unlikely & 1 \\
  Unlikely & 2 \\
  Likely & 3 \\
  Fairly likely & 4 \\
  Very likely & 5 \\
  \bottomrule
\end{tabular}
\end{minipage}
\begin{minipage}[b]{0.5\linewidth}
\begin{tabular}[c]{lc}
  \toprule
  \textbf{Severity} & \textbf{Severity rating} \\
  \midrule
  First Aid injury/illness & 1 \\
  Minor injury/illness & 2 \\
  `3 day' injury/illness & 3 \\
  Major injury/illness & 4 \\
  Fatality/disabling injury & 5 \\
  \bottomrule
\end{tabular}
\end{minipage}
\bigskip

The following should be used to rate the risk and plan corrective action:
\bigskip
\newcommand{\riskinfo}[4]{
  #1 & #2 & #3 & #4 \\
}

\begin{tabular*}{\linewidth}[c]{cccp{33em}}
  \toprule
  \textbf{Risk Rating} & \textbf{Category} & \textbf{Tolerability} & \textbf{Comments} \\
  \midrule

  \riskinfo{1--2}{Very Low}{Acceptable}
  {No further action is necessary other than to ensure that the controls are maintained.}

  \riskinfo{3--4}{Low}{Acceptable}
  {No additional controls are required unless they can be implemented at very low cost (in terms of time, money and effort).}

  \riskinfo{5--7}{Medium}{Tolerable}
  {Consideration should be given as to whether the risks can be lowered, where applicable, to a tolerable level, and preferably acceptable level, but the costs of additional risk reduction measures should be taken into account.  The risk reduction measures should be implemented within a defined time period.}

  \riskinfo{8--14}{High}{Tolerable}
  {Substantial efforts should be made to reduce the risk.  Risk reduction measures should be implemented urgently within a defined time period and it might be necessary to consider suspending or restricting the activity, or to apply interim risk control measures, until this has been completed. Considerable resources might have to be allocated to additional control measures.}

  \riskinfo{15 and above}{Very High}{Unacceptable}
  {Substantial improvements in risk control are necessary, so that risk is reduced to a tolerable or acceptable level.}

  \bottomrule
\end{tabular*}

\end{landscape}


\end{document}

