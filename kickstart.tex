\documentclass[12pt,a4paper]{scrartcl}
\usepackage{url, booktabs,pdflscape}
\usepackage[final]{pdfpages}
\usepackage[colorlinks]{hyperref}
\title{Student Robotics Kickstart Risk Assessment Form}

\begin{document}
\maketitle

\begin{description}
\item[Activity being assessed:] Student Robotics Kickstart 2013 (03/11/2012)
\item[Location:] Room 1015, Building 32, Highfield Campus, University of Southampton (\url{http://data.southampton.ac.uk/building/32.html}) and The Cube, Building 42, Highfield Campus, University of Southampton (\url{http://data.southampton.ac.uk/building/42.html})
\item[Who is exposed to the hazard:] Competitors, Team Leaders, Mentors
\end{description}

\begin{description}
\item[Assessor's name:]
\item[Assessor's job title:]
\item[Assessor's signature:]
\item[Date of assessment:]
\end{description}
\clearpage

\newcommand{\risk}[3]{
 #1 & #2 & #3 \\
}

\begin{landscape}
\section{Risks}
The following risks have been considered for the Student Robotics Kickstart event.  Further description of the meaning of risk ratings (presented in this section as $L \times S$) can be found in the next section.

A safety briefing will be given during the introductory talk, covering the points below.

\bigskip
\begin{tabular*}{\linewidth}[c]{p{14em}p{30em}c}
\toprule
\textbf{Hazard} & \textbf{Control Measures} & \textbf{Risk Rating} \\
\midrule

\risk{Injury while using mini-game tools and materials}
% Fill me in
{Student Robotics Blueshirts will supervise all use of tools and materials at
the mini-game. Since the primary material is card and the most dangerous of the
tools is scissors, Student Robotics views the risk as low. Any testing of
contraptions to be done in a Blueshirt-supervised area}
{1}

\risk{Electric shock by contact between water, electrical output and human}
{Water and electrical outputs kept strictly apart. Food and Drink is not allowed
in clearly marked areas.}
{3}

\risk{Risk of Fire}
{No naked flames are allowed to be used intentionally. If a fire breaks out
accidentally, University of Southampton regulations will be followed as detailed
below}
{2}
\bottomrule
\end{tabular*}
\end{landscape}

\input{assessment-guidance}

\clearpage
\appendix
\section{Fire Safety}
\textit{From iSolutions Regulations -- \url{http://www.southampton.ac.uk/isolutions/essentials/learnandteach/cls/fire.html}}

All Common Learning Spaces have a Fire Evacuation Route Poster located usually near the exit of in a glassed wall display cabinet alongside the other Common Learning Spaces signage.

\subsection{If you discover a fire}
\begin{enumerate}
\item Activate the alarm at any fire alarm call point by breaking the glass.
\item Evacuate the building by the most direct route.
\item Report to the assembly area
\end{enumerate}

\subsection{If you hear the alarm}
\begin{enumerate}
\item  Switch off any electrical equipment that you have been using, if safe to do so.
\item Close the door of the room when leaving.
\item Evacuate the building by the most direct route, and report to the assembly point.
\end{enumerate}

\newpage
% \includepdf[scale=1.0]{FireB32.pdf}
\includepdf[scale=1.0,landscape]{Fire1.pdf}
%\includepdf[scale=1.0,landscape]{Fire3.pdf}

\end{document}

